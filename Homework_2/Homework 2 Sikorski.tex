\documentclass[12pt]{elsarticle}

\usepackage{lineno,hyperref,notoccite,etoolbox}
\makeatletter
\def\ps@pprintTitle{%
	\let\@oddhead\@empty
	\let\@evenhead\@empty
	\def\@oddfoot{\centerline{\thepage}}%
	\let\@evenfoot\@oddfoot}
\makeatother
\usepackage{setspace}
\singlespacing
\usepackage{mathptmx}
\usepackage{float,wrapfig}
\usepackage[margin=1in]{geometry}
\usepackage{booktabs}
\usepackage{cancel}
\usepackage[fleqn]{amsmath}
\usepackage{amssymb}
\allowdisplaybreaks
\newcommand{\bnumbers}{\begin{enumerate}}
	\newcommand{\enumbers}{\end{enumerate}}
\newcommand{\vs}{\vspace{2mm}}
\newcommand{\beq}{\begin{equation*}}
\newcommand{\eeq}{\end{equation*}}
\newcommand{\rr}[1]{\mbox{#1}}
\newcommand{\longequals}{{=\joinrel=}}
\newcommand{\squared}{$^{2}$}
\newcommand{\subtwo}{$_{2}$}
%tables
\setlength{\arrayrulewidth}{0.5mm}
\setlength{\tabcolsep}{5pt}
\renewcommand{\arraystretch}{1.75}



\begin{document}
\begin{flushright}
	Ember Sikorski\par
	Homework 2\par
	ECE 624\par 
	5 October 2018
\end{flushright}


\begin{enumerate}
%1	
\item Explain why electronic doping by introduction of suitable donor/acceptor levels in an amorphous material is difficult.
\begin{itemize}
	\item \cite{Tauc1976} concentration of defect states in the gap is higher than those introduced by doping
	\item \cite{Kim2015}  Such low-coordination structure is so flexible 
	\item \cite{Fritzsche2007} VAPS make chalc glasses immune to doping because the Fermi is pinned
	\begin{itemize}
		\item doping induces structural relaxation
		\item free carriers that would be generated from dopants are now trapped
		\item resistivity remains high (VAP)
	\end{itemize}
\end{itemize}

%2
\item Explain why the doping discussed in problem 1 might not improve carrier mobility. \par \vs
\begin{itemize}
	\item \cite{Kazakova1999} trapping at negative defect centers likely limits hole drift mobility
\end{itemize}
%3
\item Would you expect to see blocking contacts in a metal-amorphous semiconductor contact? Explain your answer.

%4
\item Describe the structural order giving rise to extended (delocalized) states and localized states in an amorphous semiconductor.

%5
\item Provide definitions/explanations of the following:
\begin{itemize}
	\item Hubbard correlation energy
	\item Localization length
	\item Anderson transition
	\item Localized ``in the Anderson sense"
\end{itemize}

%6
\item \begin{enumerate}
	\item What is the difference between variable range hopping and nearest neighbor hopping?
	\item In which type of conduction will tunneling occur?
\end{enumerate}

\end{enumerate}

\section*{References}
\bibliography{homework2}
\bibliographystyle{elsarticle-num}


\end{document}  