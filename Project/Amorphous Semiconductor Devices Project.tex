\documentclass[3p,review,12pt]{elsarticle}
\usepackage{lineno,hyperref,notoccite,etoolbox}
\modulolinenumbers[5]
\makeatletter
\def\ps@pprintTitle{%
	\let\@oddhead\@empty
	\let\@evenhead\@empty
	\def\@oddfoot{\centerline{\thepage}}%
	\let\@evenfoot\@oddfoot}
\makeatother
\usepackage{setspace}
\singlespacing
\usepackage{mathptmx}
\usepackage{float,wrapfig}
\begin{document}

\begin{frontmatter}
	\title{Computational Methods for Amorphous Semiconductor Devices}
	
	\author[boise]{Ember L. Sikorski}
	
	
	\address[boise]{Boise State University}
	
	\begin{abstract}
\begin{itemize}
	\item DOS
	\item structural modeling
\end{itemize}
	\end{abstract}
	
	
\end{frontmatter}

\section{Introduction}

\section{Density of States}
\underline{Raty 2015 \cite{Raty2015}} - Aging in Phase Change Materials (dots figure)
\par
\begin{itemize}
	\item Motivation
	\begin{itemize}
		\item "Amorphous materials are out of thermodynamic equilibrium"
		\item subject to physical aging
		\item phase-change materials (PCMs) have a fast, reversible switch between a conductive crystalline and more resistive amorphous phase
		\item aging increases the resistivity - `resistance drift'
		\item computer simulation to investigate relaxation processes
		\item \textbf{Modeling comment:} complexity of the chemistry requires DFT to describe and understand bonding and the amorphous phase
	\end{itemize}
	\item Literature
	\begin{itemize}
		\item DFT simulations of GeSbTe alloys report many tetrahedrally bonded Ge, which does not exist in crystal. These are obtained from MQ calcs
	\end{itemize}
	\item Methods
	\begin{itemize}
		\item Car-Parrinello
		\item \textbf{To circumvent time scale problem, generated collection of a-structures}
		\item mixed Gaussian/plane wave code in CP2K
		\item cutoff 300 Ry
		\item sampled at gamma only
		\item annealed using plane-wave code in Quantum Espresso
		\item 34 Ry
		\item 3.84 fs
		\item Berendsen thermostat
		\item 10 models produced starting from liquid
	\end{itemize}	
 	\item Results
 	\begin{itemize}
 		\item Ge$^{T}$ is associated with homopolar Ge-Ge bonds
 		\item heat of formaion shows homopolar bonds more favorable in GeTe than GeSe and SnTe
 		\item wanted to investigate effects of varying amounts Ge-Ge bonds
 		\item used different alloys along the phase diagram and substituted with Ge or Te to form different GeTe structures "mimicking aging"
 		\item homopolar bonds correlated with tetrahedral Ge
 		\item freezing at density of amorphous GeTe, tetrahedral rich models had the largest values of stress
 		\item this agrees with experiments showing the drift of PCMS is accompanied by stress relief
 		\item order parameter $d_{4}/d_{0}$ goes from tetrahedrally bonded Ge, $Ge^{T}$, to $Ge^{III}$ and $d_{3}/d_{0}$ goes from $Te^{II}$ to $Te^{III}$
 		\item increase in band gap directly linked to decrease in homopolar bonds
 		\item "melt-quenched model has a smaller band gap and possesses a (localized) mid-gap state"
 	\end{itemize}	
\end{itemize}
\section{Density Functional Theory}
\begin{itemize}
	\item Kohn Sham: A \emph{system} of one-electrons
	\item Hartree: a \emph{potential} of how each electrons feels the electron gas
	\item Hartree Fock: how we describe the wave functions
\end{itemize}
\section*{References}

\bibliography{amorph}
\bibliographystyle{elsarticle-num}

\end{document}  