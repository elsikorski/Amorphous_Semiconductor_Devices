\documentclass[12pt]{elsarticle}

\usepackage{lineno,hyperref,notoccite,etoolbox}
\makeatletter
\def\ps@pprintTitle{%
	\let\@oddhead\@empty
	\let\@evenhead\@empty
	\def\@oddfoot{\centerline{\thepage}}%
	\let\@evenfoot\@oddfoot}
\makeatother
\usepackage{setspace}
\singlespacing
\usepackage{mathptmx}
\usepackage{float,wrapfig}
\usepackage[margin=1in]{geometry}
\usepackage{booktabs}
\usepackage{cancel}
\usepackage[fleqn]{amsmath}
\usepackage{amssymb}
\allowdisplaybreaks
\newcommand{\bnumbers}{\begin{enumerate}}
	\newcommand{\enumbers}{\end{enumerate}}
\newcommand{\vs}{\vspace{2mm}}
\newcommand{\beq}{\begin{equation*}}
\newcommand{\eeq}{\end{equation*}}
\newcommand{\rr}[1]{\mbox{#1}}
\newcommand{\longequals}{{=\joinrel=}}
\newcommand{\squared}{$^{2}$}
\newcommand{\subtwo}{$_{2}$}
%tables
\setlength{\arrayrulewidth}{0.5mm}
\setlength{\tabcolsep}{5pt}
\renewcommand{\arraystretch}{1.75}
\newcommand{\fullline}{\noindent\rule{14cm}{0.4pt} \vspace{4mm}}
\usepackage{subfigure}



\begin{document}
\begin{flushright}
	Ember Sikorski\par
	Homework 3\par
	ECE 624\par 
	19 October 2018
\end{flushright}


\begin{enumerate}
%1	
\item Describe the potential processes for optical absorption in an amorphous material, such as Ge$_{2}$Se$_{3}$. \par \vs


%2
\item Describe the processes involved in photoconductivity of a binary amorphous chalcogenide.


%3
\item Describe a Tauc plot and the application to an amorphous material. \par \vs
A Tauc plot show the transmission with respect to photon energy \cite{Zallen1983,Kim2015}. Taking the slope of the curve and finding its intercept at either 0 or 10$^{4}$ gives the optical band gap. This approach can be used to compare optical band gaps between samples, under compression, at different pressures, at different concentrations, e.g. $x$ in a-InGaZnO$_{4-x}$S$_{x}$ \cite{Kim2015}, etc.

\end{enumerate}


\section*{References}
\bibliography{homework3}
\bibliographystyle{elsarticle-num}


\end{document}  