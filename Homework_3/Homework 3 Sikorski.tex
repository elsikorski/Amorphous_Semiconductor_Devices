\documentclass[12pt]{elsarticle}

\usepackage{lineno,hyperref,notoccite,etoolbox}
\makeatletter
\def\ps@pprintTitle{%
	\let\@oddhead\@empty
	\let\@evenhead\@empty
	\def\@oddfoot{\centerline{\thepage}}%
	\let\@evenfoot\@oddfoot}
\makeatother
\usepackage{setspace}
\singlespacing
\usepackage{mathptmx}
\usepackage{float,wrapfig}
\usepackage[margin=1in]{geometry}
\usepackage{booktabs}
\usepackage{cancel}
\usepackage[fleqn]{amsmath}
\usepackage{amssymb}
\allowdisplaybreaks
\newcommand{\bnumbers}{\begin{enumerate}}
	\newcommand{\enumbers}{\end{enumerate}}
\newcommand{\vs}{\vspace{2mm}}
\newcommand{\beq}{\begin{equation*}}
\newcommand{\eeq}{\end{equation*}}
\newcommand{\rr}[1]{\mbox{#1}}
\newcommand{\longequals}{{=\joinrel=}}
\newcommand{\squared}{$^{2}$}
\newcommand{\subtwo}{$_{2}$}
%tables
\setlength{\arrayrulewidth}{0.5mm}
\setlength{\tabcolsep}{5pt}
\renewcommand{\arraystretch}{1.75}
\newcommand{\fullline}{\noindent\rule{14cm}{0.4pt} \vspace{4mm}}
\usepackage{subfigure}



\begin{document}
\begin{flushright}
	Ember Sikorski\par
	Homework 3\par
	ECE 624\par 
	19 October 2018
\end{flushright}


\begin{enumerate}
%1	
\item Describe the potential processes for optical absorption in an amorphous material, such as Ge$_{2}$Se$_{3}$. \par \vs

The largest contributions to optical absorption are transitions (i) from the valence band tail to the conduction extended states, (ii) from the valence extended states to the conduction extended states, and (iii) from the valence extended states to the conduction band tail. Numerous origins of excitation exist for each material. For instance, SiO$_{2}$ \cite{Skuja2005} has optical absorption bands from dangling O bonds, glassy disordered bonds, oxygen deficiency centers, and many more.
\vs 
%2
\item Describe the processes involved in photoconductivity of a binary amorphous chalcogenide. \par \vs

Street \cite{Street1975} uses the valence alternation pair approach to explain photoconductivity. Following this model, photoconductivity results from the excitation of holes that have recombined with a D$^{-}$ center. In order to get conduction we need long maximize exciton lifetime \cite{Mott1987} or reduce recombination rates. Street \cite{Street1975} posits the rate limiting step for recombination lies with the rate of electron tunneling between D$^{0}$ centers.

\vs
%3
\item Describe a Tauc plot and the application to an amorphous material. \par \vs
A Tauc plot show the transmission with respect to photon energy \cite{Zallen1983,Kim2015}. Taking the slope of the curve and finding its intercept at either 0 or 10$^{4}$ gives the optical band gap. This approach can be used to compare optical band gaps between samples, under compression, at different pressures, at different concentrations, e.g. $x$ in a-InGaZnO$_{4-x}$S$_{x}$ \cite{Kim2015}, etc.

\end{enumerate}


\section*{References}
\bibliography{homework3}
\bibliographystyle{elsarticle-num}


\end{document}  